\documentclass[10pt]{article}

\usepackage{url}
\usepackage{ulem}
\usepackage{geometry}
\usepackage{subfig}
\usepackage{float}
\floatstyle{boxed} 
\restylefloat{figure}

\geometry{letterpaper}

\setcounter{secnumdepth}{5}
\setcounter{tocdepth}{5}

\begin{document}

\title{\vfill Title}
\author{
 By Dirk Cummings \vspace{10pt} \\
 Orion Miller \vspace{10pt} \\ 
Senior Project  \vspace{10pt} \\ 
Dr. Philip Nico \vspace{10pt} \\ 
}
\date{December 2, 2011}

\maketitle

\vfill  %in combination with \newpage this forces the abstract to the bottom of
% the page
\begin{abstract}
Abstract
\end{abstract}

\thispagestyle{empty} %remove page number from title page

\newpage

\thispagestyle{empty}  %Remove page number from TOC
%Create a table of contents with all headings of level 3 and above.  
%http://en.wikibooks.org/wiki/LaTeX/Document_Structure#Table_of_contents has
% info on customizing the table of contents
\tableofcontents

\newpage
\setcounter{page}{1}

\section{Research of other CTFs}

\section{Objectives}

\section{The Duke Nukem Scenario}
\subsection{Objectives}
\subsection{Design}
\subsection{Game Day Results}
\subsection{Problems and Solutions}

\section{The Star Trek Scenario}
\subsection{Objectives}
After the first CTF competition, we wanted to make the next CTF focus mainly on
solving two problems:
\begin{itemize}
	\item testing the individual's coding and hacking skills rather than their
	\item ability to find and use pre-built tools and prevent the participants
	from breaking the servers (rather than the services) and ruining the game.
\end{itemize}

These would not only improve the overall game play and enjoyment, but would
help eliminate game town time spent repairing and restoring the virtual machine
server's images. Additionally, by designing new services from the ground up we
can demonstrate to participants and give them experience with the common cause
of vulnerabilities: poor design.

The first scenario was good for getting the uninitiated starting in security,
however, it did lacked While the first scenario focused mainly on attacking
servers and their services, we needed to even the importance of defending to
better align the game with a real world scenario.

Lastly, players of the first competition noted in their feedback they wanted
more teamwork based cooperation. Specifically, those with little to no security
experience had a tough time working out the solutions and would have liked to
work with others outside of their team to learn techniques and practices. Team
formation in the second competition needed to move away from the standard
method by which those who know each other tightly group together thus
separating those who know from those who don't. This method tends to create
groups with varying experience and a hording of knowledge.

\subsection{Design}
\label{stdesign}
To ensure no participant would be able to use pre-built tools to exploit
vulnerabilities, we chose to design services one might program during the first
implementations of a ship from Star Trek. This design was chosen for two
reasons: in the summer of 2011 Netflix began streaming every episode of every
Star Trek series \cite{Netflix}, and we couldn't find anyone who had created
such services or written anything to exploit such services. A listing of these
services and their designs can be found in the Game Play and Services sections.
The new approach also allowed us to make changes to the game play to focus on a
more team based play.

\subsubsection{Game Play}
\label{stgameplay}
The new game design was made to resemble a WarGame style of play consisting of
only two teams each their own server to manage. We were even fortunate enough
to provide both teams with their own rooms to work in together. Each team would
initially receive the same source code of the services to fix and deploy on
their servers (each a virtual machine running Ubuntu Server 11.10). Their goal
to find and patch the vulnerabilities in their code and exploit them in the
other team. If a team could keep their services responding and pass basic
functionality tests they would score points. However, if their services become
unresponsive or failed the tests, whether their own fault or the opposing team,
they would stop receiving points while the opposing team would score additional
points.

We also wanted to give those participants with little or no hacking, or even
coding, experience something entertaining and contributory to do. Since these
services should be fully functional and represent a star ship, then it stands
those who feel they can't contribute could fly the ship around and score points
for their team by shooting the other team's ship.

\subsubsection{Score Keeping}
\label{stscorekeeping}

\subsubsection{Services}
\label{stservices}
On competition day, there were three implemented services in production for
teams to work on with three additional services planned to be released later in
the competition. Each of the planned services and a brief description of their
functionality is listed at the end of this section.

Services are designed with two major parts, the built-in (or base)
functionality, and the competitor's implementation.

The base functionality is implemented by us and provided in unlinked
pre-compiled form for each team build from. It is meant to represent the
initial implementation of ship's services programmers. There are a few seeded
vulnerabilities and even more poor and hastily made designs. Base
implementation of a service is responsible only for:

\begin{itemize}
    \item provide only the most basic level of support and implementation
    necessary for the service to run, 
    \item setup the network support required for running the service,
    \item listen on an assigned port, accept incoming connections, and pass on
    the connection to a handler which processes the incoming data,
    \item run base functionality tests on the competitor's implementation and
    pass on the results to the Score Keeper for scoring,
    \item and keep a modest amount of internal book keeping specific to the
    service to ensure the honesty of a competitor's implementation.
\end{itemize}

While the exact functions for each service differ, every service's base
implementation has functions (defined externally in header files) for the
competitor's code to call before finishing processing an incoming request. For
each of a service's base function, there is a corresponding competitor's
wrapper function to be called before the equivalent base function. The location
of the wrapper functions are stored in a structure as function pointers defined
in a shared header file between both implementations. These wrapper functions
allow the competitors the chance to perform their own book keeping and sanitize
inputs from requests before calling the base functions. A diagram of the
typical call stack is available in Figure ~\ref{fig:wraperStructEx}. This is one
way in which the players can patch vulnerabilities and design flaws.

To communicate between and control other services, each service had a
pre-defined packet structure mimicking an IP packet. Both Teams started off with
the same structure but were allowed to add on to fulfill any future needs. The
initial structure contained only enough fields to control each specific command
of a service. Any security or checksum features were left up to the participants
to develop and deploy. 

\begin{figure}[here]
\small
\begin{verbatim}
extern struct PowerFuncs {     

   void *(*request_handler) (void *confd);

      The following functions are your wrappers around the built-in / main 
      	 Power service functions. They will be used for testing your service 
         implementation and will not be called until your write your own
         implementation. At the end of each of your implementations, you must
         call the external equivalent functions. A diagram of the setup is
         bellow along with an example of the order of the function calls.

      Diagram:

      ----------------------------------------------------------
      |  request_handler                                       |
      |                                                        |
      |     ---------------------------------------------------|
      |     |  additional_function                             |
      |     |     (if you want to add additional               |
      |     |     checking or functionality to add security)   |
      |     |                                                  |
      |     |     ---------------------------------------------|
      |     |     |  wrapper_function                          |
      |     |     |     (for your book-keeping)                |
      |     |     |     EX) pow_funcs.wadd_power(int, int)     |
      |     |     |     ---------------------------------------|
      |     |     |     |  built-in_function                   |
      |     |     |     |     EX) add_power(int, int)          |
      |     |     |     |                                      |
      ----------------------------------------------------------

      Function Call Example:
         
      --> Incoming Connection Request
            |--> Connection Accepted
            |--> pow_funcs.request_handler(connection_file_descriptor)
               |--> additional_function(...)
                  |--> pow_funcs.wadd_power(int, int)
                     |--> add_power(int, int)
                  |--> send_packet(struct PowerHeader, socket_fd)

   int (*wregister_service) (int, int);
   int (*wunregister_service) (int);
   int (*wadd_power) (int, int);
} pow_funcs;
\end{verbatim}
\caption{A 0 of the provided header file for the Power service. It
describes the structure and use of the wrapper functions as well as
demonstrating the data flow of incoming requests.}
\label{fig:wraperStructEx}
\end{figure}

\paragraph{Power} The critical service of each ship responsible for managing and
distributing a fixed amount of available ship power to all other running
services. A service may only run while it has been allocated power. Should a
service lose power, it shall be stopped and non-responsive until power is
restored. If the power service itself is shutdown or crashed, all other
services would be taken offline.

\paragraph{Engines} Just like the engines in the star ships with support for
both impulse and warp speeds. The engine service is only concerned with
managing either engine's current power allocation and speed as well as ensuring
their health. Each engine leaks varying amounts of radiation which if not
dissipated could damage the engines to the point they are no longer functional.
Scotty can fix engine damage, but he's not a bloody miracle worker; he can only
work so fast.

\paragraph{Navigation} Responsible for setting a ship's course direction,
managing the ship's internal representation of the map, and managing the
engines.

\paragraph{Communications} The communications service is independent of all
other services except for the power service, and instead of managing some
aspect of a ship, presents a team with a number of cryptography puzzles to be
decrypted, solved, and posted back to the Score Keeper for validation.

\paragraph{Weapons}

\paragraph{Shields}

\subsection{Game Day Results}
The competition started around noon on November 5th, 2011 with four participants
and three of the services in production. These were the power, engines, and
navigation services. The others had not yet been implemented, but were planned
to be release some time at the end of the first day or during the start of the
second day of competition.

Over the next three or four hours both teams were still reading through the
documentation and asking questions trying to figure out the structure of the
game and the code. Very little code was actually being written, but design flaws
and vulnerabilities were slowly being realized. One or two members of Team A
had previous experience in UNIX System Administration and immediately began
setting up monitoring tools to track incoming requests and log the commands
executed by each service. After which they moved their focus to the operating
system running their services and attempted to ``secure'' it instead of their
services. A few hours later and they abandoned this approach.

Team B on the other hand, situated in the neighboring conference room decided to
start by developing tools to help them manage the services. (The initial
implementation provided no user friendly means to control a service. A member
would have to format a packet, open a connection to the service, send the
byte stream through the connection, and process any returned information.)
Through this management tool the team was able to track command requests between
services and drop unexpected incoming requests. Given the sequence of events for
each service's commands, the team could select the authentic commands and dump
any malicious ones. This was a great start for the team as they were able to
reap quick results.

Up until this point Team A had been setting up connections and sending random
bytes of data to Team B's services. Since the original implementation was
faulty, these connections and incorrect data often crashed Team B's services,
causing them to fail tests and costing them points. When the new filter went
into place, these malicious attacks were almost completely eliminated.

Later on that night Team B started on their design for a basic level of
encryption. 

\subsection{Problems and Solutions}
\paragraph*{Lack of Experience Among Players} No one here really knows how to
exploit code, and those that do didn't participate.

For the first event we offered Security 101 sessions every Saturday leading up
to the competition to help prepare players and get them inspired. Because of
our lack of availability we weren't able to get a similar series setup. We
don't want to suggest correlation implies causation, but there was an obviously
lower attendance for the second competition.

\paragraph*{Stagnation} While one team outright dumped all incoming requests and
the other did essentially the same thing by hardening access to their services
no one not ssh'd into their server could gain control. Ultimately a good
achievement, however, game play began to stagnate.

An easy fix would be to require all services to have an open public interface.
Instead of giving teams control of the server and the ability to lock others
out, anyone should be able to control a Team's services. This does bring up the
possibility of each team's ship experiencing erratic behavior (such as rapid
navigation changes and little to no movement across the game's map) but this
can be redefined as desired behavior. Such behavior can serve as motivation for
a team to work on ways to prevent the attacks or increase the attacks on
another team to bring their ship down first.

\paragraph*{Short On Time} One quarter to design, implement, and test completely
custom services and wrap them up into a game is not enough time for only two
people. You say you're going to work on it over the break, but good luck keeping
priorities.

At least two more people and one more quarter would help prevent slipping and
bad code (see next problem).

\paragraph*{Bad Tests} We started performing our implementation tests one and a
half weeks prior to the competition. While were able to catch and solve a few
big problems before the last minute, we weren't able to perform a ``dry-run'' of
game and iron out some other major issues between the Score Keeper and the
services. A fare and functional Score Keeper is key to a successful CTF \cite{}.
Without the dry-run, we weren't able to find logic and design flaws in our
service tests and reporting which severely affected game play.

Ideally, all code (services, Score Keeper, helpers, etc.) should begin
undergoing final implementation tests and debugging four weeks out from the
competition and integration tests done two full weeks out culminating in a
complete dry-run of the game. All services and Score Keeping should be fully
implemented and tested before integration testing begins. Additionally, at least
one virtual machine image should created and saved for each of the following
(and will not be used for the competition):

\begin{itemize}
  \item a clean (without any game code) and fully configured server
  \item for each team, an working (with all game code included) instance which
  also contains and passes all implementation and integration tests
  \item for each team, the final working copy distributed at the start of
  competition
\end{itemize}

\paragraph*{Code Management} Each team's code had slightly different
configurations which had to be changed by hand within a single code base. With
10 to 12 changes to make in each service's code, insuring all changes were
properly made before each update was difficult under pressure. The design of
our Score Keeper (see Section ~\ref{stscorekeeping}) required all service
specific configurations to be contained within the pre-compiled source code
provided to each time. On a few occasions configuration settings for one team
made it into the compiled version of the other team. Resulting in services not
starting up, not communicating properly with other services, or communicating
incorrect results to the score keeper.

Ideally the services would have been written such that no updates were ever
needed to be pushed to servers during game play, however, other problems
inhibited this solution. 

Because of the way we decided to authenticate service status updates with
the proper team (as described in Section ~\ref{stscorekeeping}), we would
not be able to keep all configurations in a common header file. Instead we
should have either created two code bases (one for each team) with
configurations which don't change, or use a simple script to write all unique
configurations, compile the service, and push the update to the team's server.
The former would introduce the problem of ensuring both code bases were
synchronized with the proper updates. A simple solution, but one just as error
prone and no less irritating. The latter solution would take little time to
implement and cut update hassles down to a simple command line execution.

\paragraph*{Open-BSD} During the first competition, we had a
number of people trying to attack the server's services rather than finding the vulnerabilities
we created. Servers in the first competition were unpatched Ubuntu Server 10.10
instances with a number of open network services. Since this go around we didn't
need things like PHP, Apache, MySQL, etc., a stripped down operating system
capable of compiling code and running our network processes was ideal. After the
initial configuration, users would not have sudo-er privileges or be able to
install new packages. 

However, during our integration testing, we ran into a number of problems
getting the services to start up and behave properly. All code was POSIX
compliant but services had difficulties opening and binding sockets as well as
sending and receiving data. Limited remaining time (see \textbf{Short On Time}
and \textbf{Bad Tests}), we were not able to solve the problem and instead
switched to fully patched Ubuntu Server 11.10 instances with no sudo-er
privileges and without Apache, PHP, or MySQL installed. 

\section{Comparing Scenarios}


\section{The Next Ideal Solution}

\subsection{Game Play}

\subsection{Network}

\subsection{Services}

\subsection{Score Keeper}

\paragraph{Obfuscation}
The score keeper needs to be hidden better. Internally within the files and externally
as the score keeping server.

\textit{Internally}
Encrypted packing of the base functionallity. Would add extra prevention of people being
able to reverse engineer the base functionality.

\textit{Communication}
We had encryption and some form of basic authentication by using a password.
Needs to be better.

\textit{Externally}
Make your own virtual router to put your score keeper behind so that it makes it
more difficult to directly connect tot he score keeper but also allows you to in a central
point of score keeping see everything going on the network.
%ghetto hackerz did this

\paragraph{Scoring Metrics}
If you can generate a flag capturing means of qualifying for points it is a bit easier
[scenario one - scoring] however if the game play is similar to [game 2] then there
needs to be a better metric than just checking the up times and down times of those machines.
They definitly could be a factor in the score but shouldn't really be the majority fo the score.
create a means of hacking the server having some form of potentially layered affect on their
services so you can be more granulated or score and paussibly modify points for scoring
offensively and defensively

\subsection{Teaching, Advertisement and Timing}

\paragraph{Teaching}
Assuming the continued intent to varying skill level play and have an impact on the game
providing workshops, infosessions etc. leading up to the competition will greatly help 
improve the the average competetitors ability to effectively participate in the competition.

\paragraph{Advertise}
Spam Spam Spam and more Spam. Not literally but contininousely notify your prospective body
of people you think will potentially show up the time date and location of the competition.
Put of posters and flyers about 2 weeks before the competition date. Have  multiple
department newsletters.

\paragraph{Timing}
Being on a quarter system there is no good time to host the competition. The end of week 6 
in a quarter might be the ideal time to avoid conflicts with midterms. However, 
use best fit (least number of conflicts) when picking a date knowing no weekend you chose is ever going to be perfect.


\subsection{Suggestions}
A small list of guidelines to follow.
%Security 101 sessions the quarter of the competition. \\ \\
\begin{itemize}
\item Security workshops leading up to competition covering topics that are
  needed to know before competing.
\item Advertise every week (in conjunction with the 101 sessions).
\item Reserve Bonderson rooms WELL in advance and confirm the reservation months in
  advance.
\item The end of week 6 in a quarter might be the ideal time to avoid conflicts with
midterms. 
\item However, use best fit (least number of conflicts) when picking a date
avoiding sporting events, club functions, holidays, midterms, etc., while
knowing no weekend you chose is ever going to be perfect.\\ \\

\item No all nighters. You're mainly going to be crashing and you'd rather be up all
night during the competition rather than working on only a few hours of sleep
and heading towards 56 hours without sleep.\\ \\

\item Leave at least 2 full weeks before the start of the CTF to \textbf{finish} up
  testing and documentation.

\end{itemize}

\section{Conclusion}

\newpage
\nocite{*}
\section{Bibliography}

\bibliographystyle{IEEEannot}

\bibliography{finalpaper}

\end{document}
