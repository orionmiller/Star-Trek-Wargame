\documentclass[10pt]{article}

\usepackage{url}
\usepackage{ulem}
\usepackage{geometry}
\usepackage{float}

\geometry{letterpaper}

\setcounter{secnumdepth}{5}
\setcounter{tocdepth}{5}

\begin{document}

\title{\vfill Title}
\author{
 By Dirk Cummings \vspace{10pt} \\
 Orion Miller \vspace{10pt} \\ 
Senior Project  \vspace{10pt} \\ 
Dr. Philip Nico \vspace{10pt} \\ 
}
\date{December 2, 2011}

\maketitle

\vfill  %in combination with \newpage this forces the abstract to the bottom of the page
\begin{abstract}
Abstract
\end{abstract}

\thispagestyle{empty} %remove page number from title page
\newpage


%Create a table of contents with all headings of level 3 and above.  
%http://en.wikibooks.org/wiki/LaTeX/Document_Structure#Table_of_contents has info on customizing the table of contents
\thispagestyle{empty}  %Remove page number from TOC
\tableofcontents

\newpage
\setcounter{page}{1}

\section{Research of other CTFs}

\section{Objectives}

\section{The Doom Scenario}
\subsection{Objectives}
\subsection{Design}
\subsection{Game Day Results}
\subsection{Problems and Solutions}

\section{The Star Trek Scenario}
\subsection{Objectives}
After the first CTF competition, we wanted to make the next CTF focus mainly on
solving two problems:
\begin{itemize}
	\item testing the individual�s coding and hacking skills rather than their
	\item ability to find and use pre-built tools and prevent the participants from
	breaking the servers (rather than the services) and ruining the game.
\end{itemize}

These would not only improve the overall game play and enjoyment, but would
help eliminate game town time spent repairing and restoring the virtual machine
server�s images. Additionally, by designing new services from the ground up we
can demonstrate to participants and give them experience with the common cause
of vulnerabilities: poor design.

The first scenario was good for getting the uninitiated starting in security,
however, it did lacked While the first scenario focused mainly on attacking
servers and their services, we needed to even the importance of defending to
better align the game with a real world scenario.

Lastly, players of the first competition noted in their feedback they wanted
more teamwork based cooperation. Specifically, those with little to no security
experience had a tough time working out the solutions and would have liked to
work with others outside of their team to learn techniques and practices. Team
formation in the second competition needed to move away from the standard
method by which those who know each other tightly group together thus
separating those who know from those who don�t. This method tends to create
groups with varying experience and a hording of knowledge.

\subsection{Design}
To ensure no participant would be able to use pre-built tools to exploit
vulnerabilities, we chose to design services one might program during the first
implementations of a ship from Star Trek. This design was chosen for two
reasons: in the summer of 2011 Netflix began streaming every episode of every
Star Trek series \cite{Netflix}, and we couldn�t find anyone who had created
such services or written anything to exploit such services. A listing of these
services and their designs can be found in the Game Play and Services sections.
The new approach also allowed us to make changes to the game play to focus on a
more team based play.

\subsubsection{Game Play}
The new game design was made to resemble a WarGame style of play consisting of
only two teams each their own server to manage. We were even fortunate enough
to provide both teams with their own rooms to work in together. Each team would
initially receive the same source code of the services to fix and deploy on
their servers. Their goal to find and patch the vulnerabilities in their code
and exploit them in the other team. If a team could keep their services
responding and pass basic functionality tests they would score points. However,
if their services become unresponsive or failed the tests, whether their own
fault or the opposing team, they would stop receiving points while the opposing
team would score additional points.

We also wanted to give those participants with little or no hacking, or even
coding, experience something entertaining and contributory to do. Since these
services should be fully functional and represent a star ship, then it stands
those who feel they can�t contribute could fly the ship around and score points
for their team by shooting the other team�s ship.

\subsubsection{Score Keeping}

\subsubsection{Services}
On competition day, there were three implemented services in production for
teams to work on with three additional services planned to be released later in
the competition. Each of the planned services and a brief description of their
functionality is listed at the end of this section.

Services are designed with two major parts, the built-in (or base)
functionality, and the competitor�s implementation.

The base functionality is implemented by us and provided in unlinked
pre-compiled form for each team build from. It is meant to represent the
initial implementation of ship�s services programmers. There are a few seeded
vulnerabilities and even more poor and hastily made designs. Base
implementation of a service is responsible only for:

\begin{itemize}
    \item provide only the most basic level of support and implementation
    necessary for the service to run, 
    \item setup the network support required for running the service,
    \item listen on an assigned port, accept incoming connections, and pass on
    the connection to a handler which processes the incoming data,
    \item run base functionality tests on the competitor�s implementation and
    pass on the results to the Score Keeper for scoring,
    \item and keep a modest amount of internal book keeping specific to the
    service to ensure the honesty of a competitor�s implementation.
\end{itemize}

While the exact functions for each service differ, every service�s base
implementation has functions (defined externally in header files) for the
competitor�s code to call before finishing processing an incoming request. For
each of a service�s base function, there is a corresponding competitor�s
wrapper function to be called before the equivalent base function. The location
of the wrapper functions are stored in a structure as function pointers defined
in a shared header file between both implementations. These wrapper functions
allow the competitors the chance to perform their own book keeping and sanitize
inputs from requests before calling the base functions. A diagram of the
typical call stack is available in Diagram XX.XX. This is one way in which the
players can patch vulnerabilities and design flaws.

\paragraph{Power} The critical service of each ship responsible for managing and
distributing a fixed amount of available ship power to all other running
services. A service may only run while it has been allocated power. Should a
service lose power, it shall be stopped and non-responsive until power is
restored. If the power service itself is shutdown or crashed, all other
services would be taken offline.

\paragraph{Engines} Just like the engines in the star ships with support for
both impulse and warp speeds. The engine service is only concerned with
managing either engine�s current power allocation and speed as well as ensuring
their health. Each engine leaks varying amounts of radiation which if not
dissipated could damage the engines to the point they are no longer functional.
Scotty can fix engine damage, but he�s not a bloody miracle worker; he can only
work so fast.

\paragraph{Navigation} Responsible for setting a ship�s course direction,
managing the ship�s internal representation of the map, and managing the
engines.

\paragraph{Communicatiosn} The communications service is independent of all
other services except for the power service, and instead of managing some
aspect of a ship, presents a team with a number of cryptography puzzles to be
decrypted, solved, and posted back to the Score Keeper for validation.

\paragraph{Weapons}

\paragraph{Shields}

\subsection{Game Day Results}
\subsection{Problems and Solutions}

\section{Comparing Scenarios}

\section{The Next Ideal Solution}
\subsection{Game Play}
\subsection{Network}
\subsection{Services}
\subsection{Score Keeper}

\section{Conclusion}

\newpage
\nocite{*}
\section{Bibliography}

\bibliographystyle{IEEEannot}

\bibliography{finalpaper}

\end{document}
